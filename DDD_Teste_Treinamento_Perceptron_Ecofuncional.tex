\documentclass[
12pt,
oneside,
a4paper,
brazil
]{article}

% -------------------------------------------------
% Pacotes
% -------------------------------------------------
\usepackage[utf8]{inputenc}
\usepackage[T1]{fontenc}
\usepackage[brazil]{babel}
\usepackage{lmodern}
\usepackage{microtype}
\usepackage{geometry}
\usepackage{setspace}
\usepackage{indentfirst}
\usepackage{amsmath, amssymb}
\usepackage{hyperref}
\usepackage{enumitem}

% -------------------------------------------------
% Configurações
% -------------------------------------------------
\geometry{top=2.5cm,bottom=2.5cm,left=3cm,right=2.5cm}
\onehalfspacing
\setlength{\parindent}{1.25cm}

% -------------------------------------------------
% Título
% -------------------------------------------------
\title{\textbf{DDD — Teste e Treinamento do Perceptron Ecofuncional}\\
\large Subdomínio experimental para inferência e avaliação ecofuncional\\
\small Atualizado para v0.2.1 (features temporais de 30 entradas)}

\author{
José Pedro Trindade\\
\small Plataforma de Ecologia Computacional -- SisTerApp
}

\date{Dezembro de 2025}

% -------------------------------------------------
\begin{document}
\maketitle

% =================================================
\section{Domínio}
% =================================================

\textbf{Domínio:}  
Teste, treinamento e validação experimental de mecanismos de inferência
ecofuncional, aplicados à leitura de estados integrados de paisagens
campestres simuladas.

Este domínio não tem como objetivo substituir ou alterar os processos
ecológicos explícitos do modelo principal, mas interpretar e avaliar
estados ecofuncionais a partir de dados gerados pela simulação.

% =================================================
\section{Subdomínio e Limites de Contexto}
% =================================================

O domínio de teste e treinamento constitui um \textit{Bounded Context}
distinto do domínio ecológico principal.

\begin{itemize}
	\item \textbf{Domínio Principal}: Paisagem Ecofuncional Integrada
	\item \textbf{Subdomínio Experimental}: Inferência Ecofuncional
\end{itemize}

A comunicação entre os domínios ocorre exclusivamente por meio de dados
exportados (snapshots ecofuncionais), garantindo isolamento conceitual
e reprodutibilidade.

% =================================================
\section{Objetivo do Subdomínio}
% =================================================

\begin{itemize}
	\item Treinar modelos de inferência ecofuncional a partir de estados simulados;
	\item Testar a capacidade do perceptron em reconhecer padrões funcionais;
	\item Avaliar a coerência entre estados ecológicos explícitos e inferências;
	\item Produzir artefatos versionados (pesos) para uso em inferência.
\end{itemize}

Este subdomínio não toma decisões ecológicas nem altera estados do sistema.

% =================================================
\section{Agregado Raiz}
% =================================================

\subsection{EcofunctionalExperiment}

Representa uma execução experimental completa de teste ou treinamento.

\textbf{Identidade}
\begin{itemize}
	\item ExperimentID
\end{itemize}

\textbf{Responsabilidades}
\begin{itemize}
	\item Definir o objetivo do experimento (treino ou teste);
	\item Referenciar conjunto de dados ecofuncionais;
	\item Definir parâmetros experimentais;
	\item Produzir resultados e artefatos versionados.
\end{itemize}

% =================================================
\section{Entidades}
% =================================================

\subsection{EcofunctionalSample}

Representa uma amostra ecofuncional individual.

\textbf{Atributos}
\begin{itemize}
	\item inputVector (estado ecofuncional)
\item targetLabel (interpretação funcional esperada)
\end{itemize}

Cada amostra corresponde a uma leitura instantânea de uma célula ou
patch da paisagem simulada.

\subsection{EcofunctionalTrajectory}
Representa a sequência temporal de amostras ecofuncionais ($t_0, t_1, \dots, t_n$) para capturar histerese e momentum ecológico.

\textbf{Responsabilidades}
\begin{itemize}
	\item Armazenar histórico ordenado de \textit{EcofunctionalSample};
	\item Expor métricas derivadas: delta (derivada discreta), média móvel;
	\item Fornecer sinais de tendência (vegetação, hidrologia) usados na inferência.
\end{itemize}

% =================================================
\section{Value Objects}
% =================================================

\subsection{EcofunctionalVector}

\begin{itemize}
	\item soilDepth
	\item soilCompaction
	\item soilInfiltration
	\item hydroFlux
	\item erosionRisk
	\item vegetationCoverageEI
	\item vegetationCoverageES
	\item vegetationVigorEI
	\item vegetationVigorES
\item propagulePotential
\end{itemize}

Este vetor é imutável e representa o estado ecofuncional observado.

\subsection{FeatureVector Temporal (30 entradas)}

Concatenação de:
\begin{itemize}
	\item Estado atual (10 atributos);
	\item Delta entre amostras consecutivas (10 atributos);
	\item Média móvel (janela 3) dos mesmos atributos (10 atributos).
\end{itemize}

É o vetor efetivamente consumido pelo Perceptron.

\subsection{InferenceOutput}

\begin{itemize}
	\item resiliencePotential
	\item functionalIntegrity
	\item recoveryCapacity
\end{itemize}

Representa a saída interpretativa do perceptron.

% =================================================
\section{Serviços de Domínio}
% =================================================

\subsection{PerceptronTrainingService}

Responsável por ajustar os pesos do perceptron a partir de um conjunto
de amostras ecofuncionais, gerando o vetor temporal de 30 entradas a
partir da trajetória acumulada.

\begin{itemize}
	\item train(trajectory, parameters)
	\item validate(dataset)
\end{itemize}

\subsection{PerceptronInferenceService}

Responsável por executar inferência ecofuncional utilizando pesos
previamente treinados, consumindo o vetor de 30 entradas derivado da
trajetória para capturar tendências.

\begin{itemize}
	\item infer(trajectory)
\end{itemize}

Esses serviços não têm acesso direto ao domínio ecológico principal.

% =================================================
\section{Artefatos de Domínio}
% =================================================

\subsection{PerceptronModel}

Artefato versionado contendo:

\begin{itemize}
	\item pesos;
	\item bias;
	\item definição explícita das entradas;
	\item versão e metadados do experimento.
\end{itemize}

O modelo é tratado como produto científico reprodutível.

% =================================================
\section{Regras e Invariantes}
% =================================================

\begin{enumerate}
	\item O treinamento nunca ocorre em tempo de simulação;
	\item Pesos não são ajustados durante inferência;
	\item Inferência não altera estados ecológicos;
\item A resiliência não é inferida diretamente;
\item O experimento é sempre reprodutível a partir dos dados e parâmetros.
\item CSVs devem possuir 10 atributos + alvo (treino) ou 10 atributos (trajetória); a ingestão falha caso o arquivo não seja aberto ou tenha colunas insuficientes.
\end{enumerate}

% =================================================
\section{Relação com a Resiliência Ecológica}
% =================================================

A resiliência, conforme definida por Holling, não é uma saída direta
do perceptron nem um atributo do experimento. Ela emerge da avaliação
temporal das inferências ecofuncionais aplicadas a trajetórias do sistema.

O subdomínio experimental contribui fornecendo sinais locais e
interpretações funcionais que subsidiam essa avaliação emergente.

% =================================================
\section{Síntese}
% =================================================

Este DDD define formalmente o subdomínio de teste e treinamento do
perceptron ecofuncional como um espaço experimental isolado, coerente
com o domínio ecológico principal e alinhado à definição clássica de
resiliência ecológica. A separação entre simulação, inferência e
avaliação garante rigor científico, transparência metodológica e
extensibilidade da plataforma.

\end{document}
